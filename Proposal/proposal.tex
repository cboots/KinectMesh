\documentclass[english]{article}

\title{Thesis Proposal \\ Real-Time Generation of Independent Dynamic Triangle Mesh Models from Depth Camera Data}
\author{Collin Boots}
\date{Sept 3, 2013}

\begin{document}
\maketitle
\section*{Background}
As robots continue to be incorporated into human environments, the need for intelligent and high-speed reasoning about the objects around them increases dramatically. At the simplest level, mobile robots need to create a map of their environment for navigation. At a higher level, some robots need to recognize distinct objects in their environment, track object movement, and have some intuitive sense of object geometry that is easily stored and processed. Even more important is being able to efficiently generate this environment from sensor data in real time. Like the human brain, the robot should also be able to perform these low level functions with only minimal intervention from higher cognitive functions.\\
\\
Previous work has demonstrated the diverse capabilities of depth cameras from generating highly accurate 3D surface models \cite{KinectFusion} to reliable 3D pose estimation \cite{Endres,Taguchi}. However, these efforts have collectively suffured from several limitations. Many algorithms attempt to store the generated environment as a RGB 3D point cloud (either pruned or complete), which is not easily adaptable to dynamic environments, requires exorbitant quantities of memory to store large environments, and provides no intuition to higher perception processes about distinct objects beyond a volumetric approximation. Other approaches have been able to store and merge the surface data more efficiently, but still regard the environement as a unified whole rather than discrete objects. These approaches may be sufficient for mapping static envirnoments, but provide only limited utility to more interactive robots and can actually hinder robots that reconfigure their surroundings (and hence invalidate part of their map).

\section*{Proposed Method}


\section*{Potential Outcomes}

\bibliographystyle{plain}
\bibliography{Sources}
\end{document}
